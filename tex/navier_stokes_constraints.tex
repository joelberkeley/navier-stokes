\documentclass[10pt,a4paper]{article}
\usepackage[utf8]{inputenc}
\usepackage{amsmath}
\usepackage{amsfonts}
\usepackage{amssymb}


\newcommand{\ud}{\mathrm{d}}
\DeclareMathOperator{\tr}{tr}

\title{Constraints on an incompressible Navier-Stokes fluid}
\date{}




\begin{document}

\maketitle

The $(d+1)$-dimensional Navier-Stokes equations and the incompressibility condition are\footnote{In all calculations in this work we will assume the velocity satisfies incompressibility.}
\begin{align}
	\partial_t v_i + \partial_i P + v_j\partial_j v_i - \eta \partial^2 v_i & = 0
\\	\partial_i v_i & = 0		\quad.
\end{align}
Assuming incompressibility in all following calculations, the Navier-Stokes equations can be written as
\begin{align}
	x_i \partial_t v_j + \delta_{ij} P + v_j v_i - \eta \partial_j v_i & = f_{ij}
\\	\partial_j f_{ij} & = 0\quad.
\end{align}
Of course, we can add arbitrary terms to $f_{ij}$ whose divergence vanishes, for example
\begin{equation}
	\partial_i v_j = 0		\quad.
\end{equation}
On this basis, we will consider the equation
\begin{align}
	x_i \partial_t v_j + \delta_{ij} P + v_j v_i - \eta \partial_j v_i + \xi \partial_i v_j & = f_{ij}		\label{NS equations in terms of divergence-free tensor (E_ij)}
\\	\partial_j f_{ij} & = 0\quad.
\end{align}
where $\xi$ is an arbitrary constant.

We shall proceed by analysing the antisymmetric part and the trace of \eqref{NS equations in terms of divergence-free tensor (E_ij)}.

\section{The antisymmetric part}

The antisymmetric part of \eqref{NS equations in terms of divergence-free tensor (E_ij)} is
\begin{equation}
\partial_t v_{[j} x_{i]} - (\eta + \xi) \partial_{[j} v_{i]} = f_{[ij]}		\quad.
\label{antisymmetric part of NS}
\end{equation}
There are two cases to consider

\subsection{Case 1: $\eta + \xi = 0$}

In the case
\begin{equation}
\eta + \xi = 0\quad,
\end{equation}
equation \eqref{antisymmetric part of NS} becomes
\begin{equation}
\partial_t v_{[j} x_{i]} = f_{[ij]}		\quad,
\label{antisymmetric part for eta + xi = 0}
\end{equation}
which has solutions only if\footnote{necessary, but is it sufficient?}
\begin{equation}
x_{[i} f_{jk]} = 0    \quad.
\end{equation}
Contraction with $x_i$ gives\footnote{we have yet to show that \eqref{solution to contr. of antisym with x} solves \eqref{antisymmetric part for eta + xi = 0} - i.e.\ that they are equivalent}
\begin{equation}
v_j	=	\frac{1}{x^2}
		\left(
			2 \int \ud t \, x_i f_{[ij]} + x_j C(t,x^k)
		\right)\quad,
\label{solution to contr. of antisym with x}
\end{equation}
where we have used the fact that $x_i v_i$ is unconstrained by \eqref{antisymmetric part for eta + xi = 0} and so can therefore be replaced by an arbitrary function $C(t,x^i)$.\footnote{Note that this equation yields $h = C$. Does this have any implications for the form of $h$? Does it imply anything else?}

\subsection{Case 2: $\eta + \xi \neq 0$}

For
\[
	\eta + \xi \neq 0	\quad,
\]
we can transform to the coordinate frame
\begin{equation}
\tau =	(\eta + \xi) t - x^2/2
\qquad
y_i =	x_i
\end{equation}
or
\begin{equation}
t = \frac{\tau + y^2/2}{\eta + \xi}
\qquad
x_i = y_i \quad,
\end{equation}
where
\begin{align}
\left. \frac{\partial }{\partial \tau} \right)_{y_i}
	& = \left( \frac{\partial x_j}{\partial \tau} \right)_{y_i} \left. \frac{\partial }{\partial x_j} \right)_{t, x_{k\neq j}} + \left( \frac{\partial t}{\partial \tau} \right)_{y_i} \left. \frac{\partial }{\partial t} \right)_{x_j}
\\	& = 0 + \frac{1}{\eta+\xi} \left. \frac{\partial }{\partial t} \right)_{x_j}
\\
\frac{\partial}{\partial y_i} & = \frac{1}{\eta + \xi} x_i\partial_t + \partial_i
\end{align}

\subsubsection{Case 2i: $\eta - \xi = 0$}

For
\begin{equation}
\eta - \xi = 0
\end{equation}
the symmetric part of \eqref{} is
\begin{equation}
\partial_t v_{(j} x_{i)} + \delta_{ij} P + v_j v_i = f_{(ij)} \quad.
\label{symmetric part of E_ij}
\end{equation}
With test function
\begin{equation}
x_i v_i = h(t, x_i) \quad,
\end{equation}
we can consider the contraction of \eqref{symmetric part of E_ij} with $x^i$:
\begin{equation}
\frac{1}{2} \left(x^2 \partial_t v_j + x_j \partial_t h \right) + x_j P + v_j h = x_i f_{(ij)} \quad,
\end{equation}
or, rearranging, we find
\begin{equation}
e^{2h/x^2}\left( \frac{1}{2} x^2 \partial_t + h \right) v_j = e^{2h/x^2} \left( x_i f_{(ij)} - x_j P - \frac{1}{2} x_j \partial_t h \right) \quad,
\end{equation}
which can be rearranged as
\begin{equation}
\frac{1}{2} x^2 \partial_t \left( e^{2h/x^2} v_j \right) = e^{2h/x^2} \left( x_i f_{(ij)} - x_j P - \frac{1}{2} x_j \partial_t h \right) \quad,
\end{equation}
and integrated to give
\begin{equation}
v_j = \frac{2}{x^2} e^{-2h/x^2} \int \ud t \, e^{2h/x^2} \left( x_i f_{(ij)} - x_j P - \frac{1}{2} x_j \partial_t h \right) \quad,
\end{equation}
and integrating the last term
\begin{equation}
v_j = \frac{2}{x^2} \left[ e^{-2h/x^2} \int \ud t \, e^{2h/x^2} \left( x_i f_{(ij)} - x_j P \right) \right] - \frac{1}{2} x_j \quad.
\label{velocity from symmetric}
\end{equation}

All remaining parts of \eqref{symmetric part of E_ij} are given by contraction with $\gamma_{ik}\gamma_{jl}$, where
\begin{equation}
\gamma_{ij} = \delta_{ij} - \frac{1}{x^2}x_i x_j
\end{equation}
as
\begin{equation}
\hat v_k \hat v_l = \hat f_{(kl)} - \gamma_{kl} P
\label{projected symmetric}
\end{equation}
where $\hat{}$ implies contraction with $\gamma$. Note that this has no implications for the form of $h$, as $x$ annihilates all terms in this equation. Note that the trace of this equation is
\begin{equation}
v^2 = \gamma_{kl} f_{kl} - P(d - 1) +  \frac{h^2}{x^2}
\end{equation}

Moreover, contracting \eqref{velocity from symmetric} with $\gamma_{jk}$ gives
\begin{equation}
\hat v_k = 2x^2 x_i \gamma_{jk} e^{-2h/x^2} \int \ud t \, e^{2h/x^2} f_{(ij)}
\end{equation}
which we can compare with \eqref{projected symmetric} to find a constraint equation in $\hat f_{(ij)}$ and $P$. We can define
\begin{equation}
\sigma_{ij} = e^{-2h/x^2} \gamma_{ij}
\end{equation}
which we can use to bring this constraint into the form
\begin{equation}
P \sigma_{ij} = -4x^4 \int \ud t \, x_k f_{(kl)} \sigma_{li} \int \ud t \, x_m f_{(mn)} \sigma_{nj} + \sigma_{ik} f_{(kl)} \sigma_{lj}
\end{equation}

\section{The trace}

The trace of \eqref{NS equations in terms of divergence-free tensor (E_ij)} is
\begin{equation}
\partial_t v_i x_i + Pd + v^2 = \tr f
\label{trace of E_ij}
\end{equation}
We can solve these equations using a test function
\[
	x_i v_i = h(t,x^i).
\]

Alternatively, we can write \eqref{trace of E_ij} as
\begin{equation}
D_i D_i 1 = tr(f) - Pd
\qquad
\text{where}
\quad
D_i = x_i \partial_t + v_i
\end{equation}

\subsection{The case $x_i\partial_t v_i = 0$}
In the case
\begin{equation}
	x_i\partial_t v_i = h = 0		,
\end{equation}
equation \eqref{} becomes
\begin{equation}
	\ldots
\end{equation}




\section{Stationary incompressible Navier-Stokes}

For stationary fluids with
\begin{equation}
  \partial_t v_j = 0
\end{equation}
we have
\begin{align}
  \partial_i v_i  &=  0  \\
  v_j\partial_j v_i + \partial_i P - \eta\partial^2 v_i  &=  0
\end{align}
or, with redefining the fields such that $v_i \rightarrow \eta v_i$, $P \rightarrow \eta P$, where we effectively scale $\eta$ to 1,
\begin{align}
  \partial_i v_i  &=  0  \\
  v_j v_i - \partial_j v_i + \xi \partial_i v_j &=  f_{ij} - \delta_{ij} P  \\
  \partial_j f_{ij}  &=  0
\end{align}
Consider the following cases.

\subsection{Case 1: $\xi = 1$}
Here, we have
\begin{align}
  \partial_i v_i  &=  0  \\
  v_j v_i - 2\partial_{[j} v_{i]} &=  f_{ij} - \delta_{ij} P  \\
  \partial_j f_{ij}  &=  0
\end{align}
or
\begin{align}
  \partial_i v_i  &=  0  \\
  v_j v_i  &=  f_{(ij)} - \delta_{ij} P  \\
  2\partial_{[j} v_{i]} &=  f_{[ij]}  \\
  \partial_j f_{ij}  &=  0
\end{align}

\subsection{Case 2: $\xi = -1$}
Here we have
\begin{align}
  \partial_i v_i                      &=  0                          \\
  v_j v_i - 2\partial_{(j} v_{i)} &=  f_{(ij)} - \delta_{ij} P       \label{symmetric with xi = -1}\\
  0                                   &=  f_{[ij]}                   \\
  \partial_j f_{ij}                   &=  0
\end{align}
We can express \eqref{symmetric with xi = -1} as
\begin{equation}
  D_{(i} D_{j)} 1 = g_{ij}
\end{equation}
where
\begin{equation}
  D_i  \equiv  \partial_i - v_i    \qquad    g_{ij}  \equiv  f_{(ij)} - \delta_{ij} P
\end{equation}
or we can use $\partial_i k = v_i/2$ to write
\begin{equation}
  -2 \partial_{(i} ( e^{-k} v_{j)} )  =  e^{-k} g_{ij}
\end{equation}

\appendix

\section{Next steps}

\begin{itemize}
\item Solve incompressibility and find its implications for $h$.
\item Solve $v_j$ equations for this form of $h$
\item Figure out what to do with the hatted equation.
\end{itemize}

\section{Solution of divergence equation}

For tensor $f(t,x_k)_{ij}$ satisfying
\begin{equation}
\partial_j f_{ij} = 0		,
\end{equation}
we can use 
\begin{equation}
f(t,x_l)_{ij} = \int_{-\infty}^\infty \frac{\ud^d q}{(2\pi)^{d/2}} e^{-iq_k x_k} \tilde f(t,q_l)_{ij}
\end{equation}
to find, by using the inverse Fourier transform,
\begin{equation}
q_j \tilde f_{ij} = 0		.
\end{equation}
This equation can be solved by introducing a field
\begin{equation}
\tilde g_{ij} \equiv \tilde f_{ij} - q_j q_k \tilde h_{ik}
\end{equation}
for some arbitrary $\tilde h_{ik}$, then we find
\begin{equation}
0 = q_j \tilde f_{ij} = q_j \tilde g_{ij} + q^2q_k \tilde h_{ik}
\end{equation}
thus
\begin{equation}
q_k \tilde h_{ik} = - \frac{1}{q^2} q_j \tilde g_{ij} \qquad q_i \neq 0\quad\forall i
\end{equation}
where we have used that given $q^2 \geq 0$, having $q^2 = 0$ implies $q_i = 0 \quad\forall i$, where the previous equation is trivially satisfied. Therefore,
\begin{equation}
\tilde f_{ij} = g_{ij} - \frac{1}{q^2} q_j q_k \tilde g_{ik} = \left( \delta_{jk} - \frac{1}{q^2} q_j q_k \right) \tilde g_{ik} \qquad .
\end{equation}
taking the Fourier transform of this, we find
\begin{align}
f_{ij} 	& = \int_{-\infty}^\infty \frac{\ud^d q}{(2\pi)^{d/2}} e^{-iq_k x_k} \left( \delta_{jk} - \frac{1}{q^2} q_j q_k \right) \tilde g_{ik}
\\		& = \int_{-\infty}^\infty \frac{\ud^d q}{(2\pi)^{d/2}} e^{-iq_k x_k} \left( q^2 \delta_{jk} - q_j q_k \right) \frac{1}{q^2} \tilde g_{ik}
\\		& = \left( \delta_{jk} \partial^2 - \partial_j \partial_k \right) \int_{-\infty}^\infty \frac{\ud^d q}{(2\pi)^{d/2}} e^{-iq_k x_k}  \frac{1}{q^2} \tilde g_{ik}
\\		& = \left( \delta_{jk} \partial^2 - \partial_j \partial_k \right) h_{ik}
\end{align}
for arbitrary(?) $h_{ij}$.

\end{document}